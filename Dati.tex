%----------------------------------------------------------------------------------------
%   USEFUL COMMANDS
%----------------------------------------------------------------------------------------

\newcommand{\dipartimento}{Dipartimento di Matematica ``Tullio Levi-Civita''}

%----------------------------------------------------------------------------------------
% 	USER DATA
%----------------------------------------------------------------------------------------

% Data di approvazione del piano da parte del tutor interno; nel formato GG Mese AAAA
% compilare inserendo al posto di GG 2 cifre per il giorno, e al posto di 
% AAAA 4 cifre per l'anno
\newcommand{\dataApprovazione}{Data}

% Dati dello Studente
\newcommand{\nomeStudente}{Matteo}
\newcommand{\cognomeStudente}{Pellanda}
\newcommand{\matricolaStudente}{1125349}
\newcommand{\emailStudente}{matteo.pellanda@studenti.unipd.it}
\newcommand{\telStudente}{+ 39 346 04 01 857}

% Dati del Tutor Aziendale
\newcommand{\nomeTutorAziendale}{Erika}
\newcommand{\cognomeTutorAziendale}{Gili}
\newcommand{\emailTutorAziendale}{erika.gili@crispybacon.it}
\newcommand{\telTutorAziendale}{}
\newcommand{\ruoloTutorAziendale}{}

% Dati dell'Azienda
\newcommand{\ragioneSocAzienda}{Crispy Bacon srl}
\newcommand{\indirizzoAzienda}{Via Col. Scremin 5, Marostica (VI)}
\newcommand{\sitoAzienda}{https://crispybacon.it/}
\newcommand{\emailAzienda}{info@crispybacon.it}
\newcommand{\partitaIVAAzienda}{P.IVA 03818980249 REA VI-356597}

% Dati del Tutor Interno (Docente)
\newcommand{\titoloTutorInterno}{Prof.}
\newcommand{\nomeTutorInterno}{Paolo}
\newcommand{\cognomeTutorInterno}{Baldan}

\newcommand{\prospettoSettimanale}{
     % Personalizzare indicando in lista, i vari task settimana per settimana
     % sostituire a XX il totale ore della settimana
    \begin{itemize}
        \item \textbf{Prima Settimana - Convolgimento(40 ore)}
        \begin{itemize}
            \item Incontro con persone coinvolte nel progetto per discutere i requisiti e le richieste
            relativamente al sistema da sviluppare;
            \item Verifica credenziali e strumenti di lavoro assegnati;
            \item Presa visione dell’infrastruttura esistente;
            \item Formazione sulle tecnologie adottate;
        \end{itemize}
        \item \textbf{Seconda Settimana - Analisi (40 ore)} 
        \begin{itemize}
            \item Analisi dei requisti del prodotto;
            \item Analisi delle scelte tecnologiche;
            \item Analisi e spiegazione della scelta conversazionale;
            \item Documentazione dei risultati di analisi;
        \end{itemize}
        \item \textbf{Terza Settimana - Modello e Pianificazione (40 ore)} 
        \begin{itemize}
            \item Scelta del modello di sviluppo;
            \item Pianificazione dello sviluppo;
            \item Scelte e attuazione di repository;
            \item Inizio stesura guidata del codice;
        \end{itemize}
        \item \textbf{Quarta Settimana - Sviluppo del prodotto (40 ore)} 
        \begin{itemize}
            \item Stesura e realizzazione base della Skill;
            \item Stesura e realizzazione base dell'interfaccia conversazionale;
        \end{itemize}
        \item \textbf{Quinta Settimana - Miglioramento e Integrazione (40 ore)} 
        \begin{itemize}
            \item Miglioramento dell'interfaccia conversazionale;
            \item Realizzazione o interazione di servizi di gestione;
        \end{itemize}
        \item \textbf{Sesta Settimana - Analisi e Test (40 ore)} 
        \begin{itemize}
            \item Realizzazione e attuazione di test di sistema sull'interfaccia conversazionale;
            \item Realizzazione e attuazione di test di sistema sui servizi di gestione;
            \item Analisi dei risultati ottenuti dai test di sistema;
        \end{itemize}
    \newpage
        \item \textbf{Settima Settimana - Miglioramento e Collaudo (40 ore)} 
        \begin{itemize}
            \item Miglioramento dell'interfaccia conversazionale a fronte dei test di sistema;
			\item Miglioramento dell'interazione dei servizi di gestione a fronte dei test di sistema;
            \item Collaudo;
        \end{itemize}
        \item \textbf{Ottava Settimana - Conclusione (20 ore)} 
        \begin{itemize}
            \item Stesura documentazione finale;
            \item Incontro di presentazione della piattaforma e Live demo con gli stakeholders.
        \end{itemize}
    \end{itemize}
}

% Indicare il totale complessivo (deve essere compreso tra le 300 e le 320 ore)
\newcommand{\totaleOre}{300}

\newcommand{\obiettiviObbligatori}{
	 \item \underline{\textit{O01}}: Realizzazione interfaccia conversazionale completa descritta in fase di analisi con eventuale servizio di gestione simulato.
}

\newcommand{\obiettiviDesiderabili}{
	 \item \underline{\textit{D01}}: Realizzazione del servizio di gestione compresa l'interazione con un database.
}

\newcommand{\obiettiviFacoltativi}{
	 \item \underline{\textit{F01}}: Orchestrazione con i servizi esterni (per esempio, invio notifiche sms).
}